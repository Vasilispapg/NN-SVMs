
\documentclass{article}
\usepackage[utf8]{inputenc}
\usepackage{amsmath}
\usepackage{graphicx}
\usepackage{booktabs}
\usepackage{float}
\usepackage{array}
\usepackage[table,xcdraw]{xcolor}
\usepackage{sectsty}
\usepackage{hyperref}
\usepackage{caption}

\sectionfont{\centering}
\subsectionfont{\flushleft}

\title{\Huge SVC Classification Report}
\author{Papagrigoriou Vasileios Savvas}
\date{09-11-2023}

\begin{document}

\maketitle

\section*{Introduction}
This report provides the classification results for the CIFAR-10 dataset. The results are split into two sections, one detailing the performance on the first data batch, and the other considering all batches together.

\section*{Data}
\subsection*{SVC}
CIFAR-10
The Street View House Numbers

\section*{Notes}
In order to keep track of my work i used .ipynb jupiter notebook.
The notebook is available in the folder files.
The notebook contains all the code i used to produce the results in this report.
I splitted the code into sections that i used to produce the results for each section of this report.
Some variables may be re-initialized in different sections.
I tried to do the code as general as possible, so i can use it for all the attempts.

\section{Results for Data Batch 1}
This section presents the results when using only the first batch of the CIFAR-10 dataset.

\subsection*{GridSearchCV Results}
The GridSearchCV results of the first batch of the CIFAR-10 dataset was the following: C=10 and kernel=rbf.

\subsection*{Model Configuration}
\subsubsection*{General Parameters}
\begin{tabular}{|l|l|}
\hline
\textbf{Parameter} & \textbf{Value} \\ \hline
Estimator & SVC() \\ \hline
Parameter GridSearchCV & \{'C': [1, 10], 'kernel': ['linear', 'rbf']\} \\ \hline
\end{tabular}

\subsubsection*{Selected Parameters}
\begin{tabular}{|l|l|}
\hline
\textbf{Parameter} & \textbf{Value} \\ \hline
C & 10 \\ \hline
break\_ties & False \\ \hline
cache\_size & 200 \\ \hline
class\_weight & None \\ \hline
coef0 & 0.0 \\ \hline
decision\_function\_shape & 'ovr' \\ \hline
degree & 3 \\ \hline
gamma & 'scale' \\ \hline
kernel & 'rbf' \\ \hline
max\_iter & -1 \\ \hline
probability & False \\ \hline
random\_state & None \\ \hline
shrinking & True \\ \hline
tol & 0.001 \\ \hline
verbose & False \\ \hline
\end{tabular}

\subsection*{Performance Metrics}
\begin{tabular}{|l|l|}
\hline
\textbf{Metric} & \textbf{Value} \\ \hline
Accuracy & 0.4766 \\ \hline
F1 Score (Weighted Average) & 0.4762 \\ \hline
\end{tabular}

\subsection*{Detailed Classification Report}

\begin{table}[H]
\centering
\begin{tabular}{|c|c|c|c|c|}
\hline
\rowcolor[HTML]{ECF4FF} 
\textbf{Class} & \textbf{Precision} & \textbf{Recall} & \textbf{F1-Score} & \textbf{Support} \\ \hline
0 & 0.54 & 0.56 & 0.55 & 1000 \\ \hline
1 & 0.54 & 0.58 & 0.56 & 1000 \\ \hline
2 & 0.34 & 0.40 & 0.37 & 1000 \\ \hline
3 & 0.31 & 0.33 & 0.32 & 1000 \\ \hline
4 & 0.42 & 0.39 & 0.40 & 1000 \\ \hline
5 & 0.42 & 0.35 & 0.38 & 1000 \\ \hline
6 & 0.52 & 0.52 & 0.52 & 1000 \\ \hline
7 & 0.57 & 0.50 & 0.53 & 1000 \\ \hline
8 & 0.58 & 0.64 & 0.61 & 1000 \\ \hline
9 & 0.54 & 0.52 & 0.53 & 1000 \\ \hline
\end{tabular}
\caption*{Classification Report for Data Batch 1}
\end{table}


\section{Combined Results for All Batches}
This section presents the results from the analysis of the combined data from all batches of the CIFAR-10 dataset.


\subsection*{GridSearchCV Results}
The GridSearchCV results of all the batches of the CIFAR-10 dataset was taking too long to compute, so i used only the first 1000 samples of each batch.
The GridSearchCV results of the first 1000 samples of each batch of the CIFAR-10 dataset was the following: {C = 10, kernel = rbf}

\subsection*{Model Configuration}
The results of performance metrics and classification report are for the SVC model with the selected parameters from the GridSearchCV.
So the SVC model which performed was selected with parameters: {C = 10, kernel = rbf}

\subsection*{Performance Metrics}
\begin{tabular}{|l|l|}
\hline
\textbf{Metric} & \textbf{Value} \\ \hline
Accuracy & 0.5687 \\ \hline
F1 Score (Weighted Average) & 0.5693 \\ \hline
\end{tabular}

\subsection*{Detailed Classification Report}
\begin{table}[H]
\centering
\begin{tabular}{|c|c|c|c|c|}
\hline
\rowcolor[HTML]{ECF4FF} 
\textbf{Class} & \textbf{Precision} & \textbf{Recall} & \textbf{F1-Score} & \textbf{Support} \\ \hline
           0      & 0.62   &   0.67  &    0.64   &   1000 \\ \hline
           1     &  0.65 &     0.67&      0.66&      1000 \\ \hline
           2     &  0.45 &     0.47&      0.46&      1000 \\ \hline
           3     &  0.38 &     0.41&      0.39&      1000 \\ \hline
           4      & 0.50  &    0.48&      0.49&      1000 \\ \hline
           5      & 0.50  &    0.47&      0.48&      1000 \\ \hline
           6      & 0.62  &    0.60&      0.61&      1000 \\ \hline
           7      & 0.66  &    0.59&      0.62&      1000 \\ \hline
           8      & 0.71  &    0.69&      0.70&      1000 \\ \hline
           9      & 0.62  &    0.63&      0.63&      1000 \\ \hline
\end{tabular}
\caption*{Classification Report for all Data Batches}
\end{table}

\section{Attemps For Cifar-10}
I used GridSearchCV to find the best parameters for the SVC model.
Using only the first batch of the CIFAR-10 dataset, i found that the best parameters are C = 10 and kernel = rbf.
Using all batches of the CIFAR-10 dataset, the computation time was too long, so i used only the first 1000 samples of each batch.


\section{The Street View House Numbers}



\end{document}